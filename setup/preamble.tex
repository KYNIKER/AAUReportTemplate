\documentclass[11pt,a4paper]{report}
%%%%%%%%%%%%%%%%%%%%%%%%%%%%%%%%%%%%%%%%%%%%%%%%
% Language, Encoding and Fonts
% http://en.wikibooks.org/wiki/LaTeX/Internationalization
%%%%%%%%%%%%%%%%%%%%%%%%%%%%%%%%%%%%%%%%%%%%%%%%
% Select encoding of your inputs. Depends on
% your operating system and its default input
% encoding. Typically, you should use
%   Linux  : utf8 (most modern Linux distributions)
%            latin1 
%   Windows: ansinew
%            latin1 (works in most cases)
%   Mac    : applemac
% Notice that you can manually change the input
% encoding of your files by selecting "save as"
% an select the desired input encoding. 
\usepackage[utf8]{inputenc}
% Make latex understand and use the typographic
% rules of the language used in the document.
\usepackage[english]{babel}
\usepackage{csquotes}
% Use the palatino font
\usepackage[sc]{mathpazo}
\linespread{1.05}         % Palatino needs more leading (space between lines)
% Choose the font encoding
\usepackage[T1]{fontenc}

%%%%%%%%%%%%%%%%%%%%%%%%%%%%%%%%%%%%%%%%%%%%%%%%
% Graphics and Tables
% http://en.wikibooks.org/wiki/LaTeX/Importing_Graphics
% http://en.wikibooks.org/wiki/LaTeX/Tables
% http://en.wikibooks.org/wiki/LaTeX/Colors
%%%%%%%%%%%%%%%%%%%%%%%%%%%%%%%%%%%%%%%%%%%%%%%%
% load a colour package
\usepackage{xcolor}
%AAU colors specified at https://www.design.aau.dk/om-aau-design/Farver/
%primary colors
\definecolor{aauDarkBlue}{RGB}{33,26,82}% dark blue used in logo
\definecolor{aauBlue}{RGB}{89, 79, 191}% blue
\definecolor{aauGray}{RGB}{84, 97, 110}% gray
%secondary colors
\definecolor{aauDarkOrange}{RGB}{187, 91, 23}
\definecolor{aauOrange}{RGB}{223, 142, 48}
\definecolor{aauDarkYellow}{RGB}{151, 112, 31}
\definecolor{aauYellow}{RGB}{177, 147, 53}
\definecolor{aauDarkTurquoise}{RGB}{0, 127, 163}
\definecolor{aauTurquoise}{RGB}{49, 169, 193}
\definecolor{aauDarkSienna}{RGB}{161, 101, 71}
\definecolor{aauSienna}{RGB}{204, 139, 102}
\definecolor{aauDarkGreen}{RGB}{14, 133, 99}
\definecolor{aauGreen}{RGB}{92, 175, 141}
\definecolor{aauDarkPink}{RGB}{204, 68, 91}
\definecolor{aauPink}{RGB}{231, 130, 147}

%colors to highlight keywords in lstlisting
\definecolor{codeGreen}{rgb}{0,0.6,0}
\definecolor{codeGray}{rgb}{0.5,0.5,0.5}
\definecolor{codePurple}{rgb}{0.58,0,0.82}
\definecolor{codeBack}{rgb}{0.95,0.95,0.92}

\renewcommand\fbox{\fcolorbox{aauGray}{white!0}}
% The standard graphics inclusion package
\usepackage{graphicx}
% Set up how figure and table captions are displayed
\usepackage{caption}
\captionsetup{%
  font=footnotesize,% set font size to footnotesize
  labelfont=bf % bold label (e.g., Figure 3.2) font
}
% Make the standard latex tables look so much better
\usepackage{array,booktabs}
% Enable the use of frames around, e.g., theorems
% The framed package is used in the example environment
\usepackage{framed}

% Adds support for full page background picture
%\usepackage[contents={},color=gray]{background}
%\usepackage[contents=draft,color=gray]{background}



%Mine pakker
\usepackage{float}
\usepackage{adjustbox}
\usepackage{listings}

\usepackage{tikz}
\usetikzlibrary{shapes.geometric, arrows}
\tikzstyle{startstop} = [rectangle, rounded corners, minimum width=2cm, minimum height=1cm,text centered, draw=black, fill=red!30]
\tikzstyle{io} = [trapezium, trapezium left angle=70, trapezium right angle=110, minimum width=2cm, minimum height=1cm, text centered, draw=black, fill=blue!30]
\tikzstyle{process} = [rectangle, minimum width=2cm, minimum height=1cm, text centered,text width=4cm, draw=black, fill=orange!30]
\tikzstyle{decision} = [diamond, minimum width=2cm, minimum height=1cm, text centered,text width=4cm, draw=black, fill=green!30]
\tikzstyle{arrow} = [thick,->,>=stealth]



\lstdefinestyle{mystyle}{
    backgroundcolor=\color{backcolour},   
    commentstyle=\color{codegreen},
    keywordstyle=\color{magenta},
    numberstyle=\tiny\color{codegray},
    stringstyle=\color{codepurple},
    basicstyle=\ttfamily\footnotesize,
    breakatwhitespace=false,         
    breaklines=true,                 
    captionpos=b,                    
    keepspaces=true,                 
    numbers=left,                    
    numbersep=5pt,                  
    showspaces=false,                
    showstringspaces=false,
    showtabs=false,                  
    tabsize=2
}
%\lstMakeShortInline[language=C, basicstyle=\ttfamily, keywordstyle=\color{magenta}, stringstyle=\color{codepurple}, numberstyle=\tiny\color{codegray} ]@
%
%\lstset{style=mystyle}

%%%%%%%%%%%%%%%%%%%%%%%%%%%%%%%%%%%%%%%%%%%%%%%%
% Mathematics
% http://en.wikibooks.org/wiki/LaTeX/Mathematics
%%%%%%%%%%%%%%%%%%%%%%%%%%%%%%%%%%%%%%%%%%%%%%%%
% Defines new environments such as equation,
% align and split 
\usepackage{amsmath}
% Adds new math symbols
\usepackage{amssymb}
% Use theorems in your document
% The ntheorem package is also used for the example environment
% When using thmmarks, amsmath must be an option as well. Otherwise \eqref doesn't work anymore.
\usepackage[framed,amsmath,thmmarks]{ntheorem}

%%%%%%%%%%%%%%%%%%%%%%%%%%%%%%%%%%%%%%%%%%%%%%%%
% Page Layout
% http://en.wikibooks.org/wiki/LaTeX/Page_Layout
%%%%%%%%%%%%%%%%%%%%%%%%%%%%%%%%%%%%%%%%%%%%%%%%
% Change margins, papersize, etc of the document
\usepackage{geometry}
% Modify how \chapter, \section, etc. look
% The titlesec package is very configureable
\usepackage{titlesec}
\titleformat{\chapter}[display]{\normalfont\huge\bfseries}{}{0pt}{\centering\Huge\color{aauDarkBlue}}

% Change the headers and footers
\usepackage{fancyhdr}
\pagestyle{fancy}
\fancyhf{} %delete everything
\renewcommand{\headrulewidth}{0pt} %remove the horizontal line in the header
\renewcommand{\chaptermark}[1]{\markboth{#1}{#1}}
\fancyhead[L]{\chaptername\ \thechapter\ --\ \leftmark}
\cfoot{\thepage}
%\fancyhead[LO]{\small\nouppercase\rightmark} %uneven page - section title
%\fancyhead[LE,RO]{\thepage} %page number on all pages
%\setlength{\headheight}{13.59999pt}
%% Do not stretch the content of a page. Instead,
%% insert white space at the bottom of the page
%\raggedbottom
% Enable arithmetics with length. Useful when
% typesetting the layout.
\usepackage{calc}

%%%%%%%%%%%%%%%%%%%%%%%%%%%%%%%%%%%%%%%%%%%%%%%%
% Bibliography
% http://en.wikibooks.org/wiki/LaTeX/Bibliography_Management
%%%%%%%%%%%%%%%%%%%%%%%%%%%%%%%%%%%%%%%%%%%%%%%%
%\usepackage[backend=biber,
%  style=numeric-comp
%  ]{biblatex}
%\addbibresource{bib/mybib}

\usepackage[backend=biber,style=numeric-comp,sorting=ynt]{biblatex}
\bibliography{bib/mybib}

%%%%%%%%%%%%%%%%%%%%%%%%%%%%%%%%%%%%%%%%%%%%%%%%
% Misc
%%%%%%%%%%%%%%%%%%%%%%%%%%%%%%%%%%%%%%%%%%%%%%%%
% Add bibliography and index to the table of
% contents
\usepackage[nottoc]{tocbibind}
% Add the command \pageref{LastPage} which refers to the
% page number of the last page
\usepackage{lastpage}
% Add todo notes in the margin of the document
%\usepackage[
%%  disable, %turn off todonotes
%  colorinlistoftodos, %enable a coloured square in the list of todos
%  textwidth=\marginparwidth, %set the width of the todonotes
%  textsize=scriptsize, %size of the text in the todonotes
%  ]{todonotes}

%%%%%%%%%%%%%%%%%%%%%%%%%%%%%%%%%%%%%%%%%%%%%%%%
% Hyperlinks
% http://en.wikibooks.org/wiki/LaTeX/Hyperlinks
%%%%%%%%%%%%%%%%%%%%%%%%%%%%%%%%%%%%%%%%%%%%%%%%
% Enable hyperlinks and insert info into the pdf
% file. Hypperref should be loaded as one of the 
% last packages
\usepackage{hyperref}
\hypersetup{%
	pdfpagelabels=true,%
	plainpages=false,%
	pdfauthor={Author(s)},%
	pdftitle={Title},%
	pdfsubject={Subject},%
	bookmarksnumbered=true,%
	colorlinks=false,%
	citecolor=black,%
	filecolor=black,%
	linkcolor=black,% you should probably change this to black before printing
	urlcolor=black,%
	pdfstartview=FitH%
}

\usepackage{multirow}
